
%%%%%%%%%%%%%%%%%%%%%%%%%%%%%%%%%%%%%%%%%
% Short Sectioned Assignment
% LaTeX Template
% Version 1.0 (5/5/12)
%
% This template has been downloaded from:
% http://www.LaTeXTemplates.com
%
% Original author:
% Frits Wenneker (http://www.howtotex.com)
%
% License:
% CC BY-NC-SA 3.0 (http://creativecommons.org/licenses/by-nc-sa/3.0/)
%
%%%%%%%%%%%%%%%%%%%%%%%%%%%%%%%%%%%%%%%%%

%----------------------------------------------------------------------------------------
%	PACKAGES AND OTHER DOCUMENT CONFIGURATIONS
%----------------------------------------------------------------------------------------

\documentclass[paper=a4, fontsize=11pt]{scrartcl} % A4 paper and 11pt font size

\usepackage[T1]{fontenc} % Use 8-bit encoding that has 256 glyphs
\usepackage{fourier} % Use the Adobe Utopia font for the document - comment this line to return to the LaTeX default
\usepackage[english]{babel} % English language/hyphenation
\usepackage{amsmath,amsfonts,amsthm} % Math packages

\usepackage[usenames, dvipsnames]{color} % Allows to have color :)

\usepackage{sectsty} % Allows customizing section commands
\allsectionsfont{\centering \normalfont\scshape} % Make all sections centered, the default font and small caps

\usepackage{fancyhdr} % Custom headers and footers
\pagestyle{fancyplain} % Makes all pages in the document conform to the custom headers and footers
\fancyhead{} % No page header - if you want one, create it in the same way as the footers below
\fancyfoot[L]{} % Empty left footer
\fancyfoot[C]{} % Empty center footer
\fancyfoot[R]{\thepage} % Page numbering for right footer
\renewcommand{\headrulewidth}{0pt} % Remove header underlines
\renewcommand{\footrulewidth}{0pt} % Remove footer underlines
\setlength{\headheight}{13.6pt} % Customize the height of the header

\numberwithin{equation}{section} % Number equations within sections (i.e. 1.1, 1.2, 2.1, 2.2 instead of 1, 2, 3, 4)
\numberwithin{figure}{section} % Number figures within sections (i.e. 1.1, 1.2, 2.1, 2.2 instead of 1, 2, 3, 4)
\numberwithin{table}{section} % Number tables within sections (i.e. 1.1, 1.2, 2.1, 2.2 instead of 1, 2, 3, 4)

\setlength\parindent{0pt} % Removes all indentation from paragraphs - comment this line for an assignment with lots of text

%----------------------------------------------------------------------------------------
%	TITLE SECTION
%----------------------------------------------------------------------------------------

\newcommand{\horrule}[1]{\rule{\linewidth}{#1}} % Create horizontal rule command with 1 argument of height

\title{
\normalfont \normalsize
\textit{In The Name of God} \\
\textsc{Computer Engineering Department of Amirkabir University of Technology} \\ [25pt]
\horrule{0.5pt} \\[0.4cm] % Thin top horizontal rule
\huge Embedded Systems Homework - 3 \\ % The assignment title
\horrule{2pt} \\[0.5cm] % Thick bottom horizontal rule
}

\author{Iman Tabrizian (9331032)}

\date{\normalsize\today}

\begin{document}

\maketitle

%------- P1

\section{Problem 2}

We convert the number to binary and store the part with less value inside
the little endian and store the part with greater value inside the big endian.
And continue storing the parts with more values in the successive registers.

\begin{tabular}{|c|c|c|}
    \hline
    Address & Little Endian & Big Endian \\ \hline
    N & D1 & DE \\ \hline
    N + 1 & DE & CO \\ \hline
    N + 2 & CO & DE \\ \hline
    N + 3 & DE & D1 \\ \hline
\end{tabular}

\section{Problem 3}

We have to substract from SP by 4 for every push that we make.

Push r0: $0x0000\_2220 - 0x0000\_0004 = 0x0000\_221C$ \\

Push r1: $0x0000\_221C - 0x0000\_0004 =0x0000\_2218 $ \\

\section{Problem 4}

We should load 1000 to R3 but because \textbf{MOV} can only operate on
8 bit values, we can load 250 to R3 and shift it two times.

MOV R3, 11111010 \\
LSL R3, R3 \\
LSL R3, R3 \\
SUB R1, R6, R3 \\

\section{Problem 5}

Firstly, we store addresses inside the R0 and R1 then we do the multipication
and we store the result of R2 inside the 0x2000\_0010.

LDR R0, =0X1234\_5678 \\
LDR R1, =0X7894\_5612 \\
MUL R2, R0, R1 \\
STR R2, =0X2000\_0010 \\


\section{Problem 6}

BLX does the exchange and moves instruction set from Thumb2 to arm and conversly.
BL has more speed and doesn't use a register.


\begin{center}
    \textbf{Powered by \LaTeX}
\end{center}
\end{document}
